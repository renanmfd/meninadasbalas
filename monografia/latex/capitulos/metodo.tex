\chapter{Método}

Neste capítulo descreverei as tecnologias utilizadas e o por que de sua escolha.

% =========================================================================== %
\section{Ambiente de Desenvolvimento}

TODO Host e dominio
Como dito anteriormente, temos muitas opções de servidores para instalarmos este e-commerce. Para a escolha, foi levada em consideração a recomendação dos usuários do framework que será utilizado (será explanado na próxima seção) postado em www.drupal.org/hosting, o preço oferecido, a quantidade de domínios possíveis e o espaço em disco disponibilizado. As melhores opções na época de contatação eram:

\begin{center}
  \begin{tabular}{ | l | l | l | l |}[Hosting]
    \hline
    Nome      & Espaço    & Domínios  & Valor         \hline
    Bluehost  & Ilimitado & 3         & R\$ 15,90/mês \hline
    Hostgator & Ilimitado & Ilimitado & R\$ 9,99/mês  \hline
    GoDaddy   & Ilimidado & Ilimitado & R\$ 10,99/mês \hline
    SiteGroud & 20 GB     & Ilimitado & R\$ 14,99/mês \hline
  \end{tabular}
  Tablela
\end{center}

% =========================================================================== %
\section{Back-end}

TODO Drupal
TODO Modulos Drupal
TODO Apache
TODO PHP
TODO MySQL
TODO Vagrant


% =========================================================================== %
\section{Front-end}

TODO Bootstrap
TODO Temas Drupal
TODO LESS
TODO Javascript
TODO jQuery
TODO Gulp


% =========================================================================== %
\section{Performance}

TODO Agregação e minificação
TODO Cache
TODO Métodos de medição

% =========================================================================== %
\section{SEO}

Utilizaremos algumas ferramentas para controlar e ter melhor visibilidade ao website. São estas:

\begin{itemize}
  \item Google Analytics: Serviço gratuito de monitoramento de visitas, com dados geográficos, origem do link, sistema operacional e navegador, entre outras informações \cite{Analytics}.
  \item Google Search Console: Serviço gratuíto para monitoramento de indexação do site no buscador do Google com ferramentas para melhora-la. 
  \item PageSpeed Insights: Uma ferramenta online que ajuda a identificar as melhores práticas de performance em um website e prove sugestões de otimizações.
  \item GTMetrix: Ferramenta online que analisa a velocidade de um website e da recomendações para melhora-la.
\end{itemize}

TODO Metatags
TODO Microdata
TODO Estrutura HTML
TODO Redes sociais

