\chapter{Objetivo}

% =========================================================================== %
\section{Objetivo Geral}

O objetivo deste trabalho é criar um website e-commerce para a empresa Menina das Balas de Coco. O site terá uma area de publicação de notícias, tipo blog e também cadastro de produtos do tipo bala de coco. Deverá ser possível passar por todo o fluxo de compras no site, sendo aceitável um redirecionamento na hora do pagamento, para plataforma especializada. A performace do website deverá ser medida em softwares adequados e o resultado deve ser acima da média. Os dados dos clientes devem ser captados e mantidos de forma segura.

% =========================================================================== %
\section{Objetivo Específico}

\subsection{E-commerce}

Como todo e-commerce, o site deverá ser capaz de guiar o usuário no processo de compra do produto a ser vendido. A facilidade de uso deste mecanismo é fundamental para o sucesso do negócio. Deverá ser possível para o administrador do site cadastrar o produto e todas as suas variantes de sabor, recheio e cobertura, com o preço em Reais (BRL), um texto de apresentação do produto e imagens. Vitrines destes produtos deverão estar em todas as páginas do site, principalmente na homepage. Por ser a principal razão do site, os produtos deverão ter destaque sempre que mostrados, com uma cor que contraste com o tema do site.

\subsection{Funcionalidades}

O site deverá possuir um método de postagem de notícias, informações sobre produtos ou a empresa, como um blog. O link e resumo dos mais recentes devem ser exibidos em vitrines na homepage e devem ser possível compartilha-los em redes sociais, como Facebook e Twitter. Também na homepage, em destaque no topo da página, teremos um carrosel de banners (imagem com a largura da tela) com um título e um texto curto em cima deste, para produtos e promoções em destaque.

A aplicação deverá permitir o cadastro de usuários, tanto para identificação na hora da compra, como para a postagem de opiniões sobre a empresa e review dos produtos. Tanto as opiniões quanto os reviews devem ter um sistema de filtragem, sendo publicados apenas com a permissão do administrador e também terão uma vitrine das mais recentes na página inicial.

\subsection{Performance e SEO}

Utilizaremos o máximo possível de técnicas para melhorar a performance do site. Ao final, apresentaremos os testes de performance, comparando-o com os sites mais acessados da internet no momento. É esperado que o tempo de carregamento médio do site fique abaixo dos 3 segundos para proporcionar a melhor experiência para o usuário. 

Com uma performance boa, teremos o primeiro passo dado para um boa indexação do site nos mecanismos de busca. Utilizaremos e aprenderamos sobre tecnicas como metatags, microdata, estrutura do HTML e midias sociais.

\subsection{Tecnologias}

Como objetivo final, iremos pesquisar e aprender sobre as tecnologias mais utilizadas para desenvolvimento web, o que nos ajudará a atingir todos os outros objetivos citados acima da melhor maneira possível.
