\chapter{Objetivo}

% =========================================================================== %
\section{Objetivo Geral}

O objetivo deste trabalho é criar um website e-commerce para a empresa Menina das Balas de Coco. O site terá uma area de publicação de notícias, tipo blog e também cadastro de produtos do tipo bala de coco. Deverá ser possível passar por todo o fluxo de compras no site, sendo aceitável um redirecionamento na hora do pagamento, para plataforma especializada. A performace do website deverá ser medida em softwares adequados e o resultado deve ser acima da média. Os dados dos clientes devem ser captados e mantidos de forma segura.

% =========================================================================== %
\section{Objetivo Específico}

\subsection{E-commerce}

Como todo e-commerce, o site deverá ser capaz de guiar o usuário no processo de compra do produto a ser vendido. A facilidade de uso deste mecanismo é fundamental para o sucesso do negócio. Deverá ser possível para o administrador do site cadastrar o produto e todas as suas variantes de sabor, recheio e cobertura, com o preço em Reais (BRL), um texto de apresentação do produto e imagens. Vitrines destes produtos deverão estar em todas as páginas do site, principalmente na homepage. Por ser a principal razão do site, os produtos deverão ter destaque sempre que mostrados, com uma cor que contraste com o tema do site.

\subsection{Funcionalidades}

\subsubsection{Notícias}
O site deverá possuir um método de postagem de notícias, informações sobre produtos ou a empresa, como um blog.
Este conteúdo terá um título, uma imagem e um corpo que possa conter texto, imagens e videos. Na pagina de descrição deste post, deverá ser possível compartilha-los em redes sociais, como Facebook e Twitter com simples botão. Na página inicial do site, deverá haver uma breve listagem dos posts mais recentes, somente o título e a imagem com link para uma página com o conteúdo completo.

\subsubsection{Baner}
Teremos também um baner na homepage, utilizado em estrutura carrosel. Este baner será uma imagem da largura da tela, com um título e um texto complementar curto sobrepondo esta imagem. O objetivo deste é dar destaque a qualquer tipo de promoção, postagem ou produto do site.

\subsubsection{Usuários}
A aplicação deverá permitir o cadastro de usuários, tanto para identificação na hora da compra, como para a postagem opiniôes e ingresso em lista de e-mail. Será dado como alternativa ao formulário de cadastro o registro por redes sociais, facilitando o processo. O usuário deverá prover pelo menos nome, email e estado para finalizar o cadastro. Ao iniciar o processo de checkout de uma compra, será requisitado informações adicionais que sejam necessárias, como endereço e documentos.

\subsubsection{Review}
Para dar espaço a opiniões sobre o produto para os clientes, teremos a possibilidade de ter link enviado por e-mail para opiniões sobre os produtos que forem comprados. Este e-mail será enviado pelo site e o conteúdo publicado somente com o aval do administrador. Estes conteúdos terão uma vitrine na homepage e serão mostrados nas páginas dos respectivos produtos.

\subsubsection{Opinião}
Além de opinar sobre o produto, o cliente poderá ser requisitado a dar sua opinião sobre a empresa. Esta informação poderá ser utilizada na homepage, filtrada pelo administrador.

\subsubsection{Páginas de FAQ e Contato}
Para facilitar para os clientes e potencialmente diminuir a necessidade de atendimento direto, o site deverá ter uma página com as perguntas mais frequêntes e suas respostas. Este conteúdo deverá ter fácil sistema de cadastro para o administrador e links em lugares de fácil acesso.

% =========================================================================== %
\subsection{Performance e SEO}

Utilizaremos o máximo possível de técnicas para melhorar a performance do site. Ao final, apresentaremos os testes de performance, comparando-o com os sites mais acessados da internet no momento. É esperado que o tempo de carregamento médio do site fique abaixo dos 3 segundos para proporcionar a melhor experiência para o usuário.

Com uma performance boa, teremos o primeiro passo dado para um boa indexação do site nos mecanismos de busca. Utilizaremos e aprenderamos sobre tecnicas como metatags, microdata, estrutura do HTML e midias sociais.

\subsection{Tecnologias}

Como objetivo final, iremos pesquisar e aprender sobre as tecnologias mais utilizadas para desenvolvimento web, o que nos ajudará a atingir todos os outros objetivos citados acima da melhor maneira possível.
