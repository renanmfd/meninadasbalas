\chapter{Introdução}

% =========================================================================== %
\section{O Comércio Convencional}

Quando falava-se em comércio à 20 anos atrás, pensavamos em uma estabelecimento comercial, que podiamos nos deslocar até ele, entrar e sermos atendidos por um vendedor. Um lugar onde podiamos ver o produto que queriamos comprar, tocar, analisar sua cor e sentir seu cheiro. Após escolher, pagariamos com dinheiro ou cheque e levariamos para a casa o produto.


% =========================================================================== %
\section{A Internet}

O comércio muda muito sua cara de tempos em tempos, mas nunca tivemos um salto tão grande na forma de comprar e vender mercadorias como nos últimos 20 anos. Desde 1995 quando a Amazon, gigante americana do comércio eletrônico, fez sua primeira venda online, em uma época que o Brasil ainda estava conhecendo a internet, muita coisa mudou. Novas tecnologias nos permitem comprar e vender sem sair do conforto de nossas casas e também à pagar, receber e movimentar dinheiro por dispositívos móveis que estão conectados a internet 24h por dia, sem ter que ir ao banco e enfrentar filas.

Todas estas mudanças fizeram com que as empresas que quisesem continuar no mercado e ter sucesso nos tempos atuais, mudassem sua maneira de fazer negócio. A importância do comércio físico, apesar de grande, está em decadência, e a presença na web é um requisito fundamental. Empresas que são recordistas em vendas em lojas físicas, como por exemplo a Casa Bahia e o Magazine Luiza, investiram e ainda estão investindo pequenas fortunas na criação e manutenção de seus websites e aplicativos móveis.


% =========================================================================== %
\section{E-commerce}

Comércio eletrônico, e-business ou e-commerce são termos usados para definir qualquer tipo de negociação que envolva trasmissão de dados pela internet\cite{WhatIsEcommerce}. Vários tipos de negócios se encacham nesta definição, de sites de venda de camisetas à sistemas online de bancos onde o cliente pode fazer transações e contratar serviços online.

\subsection{Vantagens}

Uma loja online permite a uma empresa vender produtos com um preço diferenciado, beneficiando os compradores. Isso acontece pois seus gastos são menores, principalmente com vendedores e espaços físicos.

No modo tradicional de comércio, uma empresa que queira ter grande visibilidade, tem que desembolsar grandes quantidades de capital para se intalar em lugares estratégicos, onde o fluxo de compradores é grande. Na internet, um site e-commerce de uma empresa pequena tem tantas chances quanto o de uma empresa com maior capacidade financeira. Tudo depende da qualidade e segurança do site e da força do marketing. Com muito pouco capital, faz-se campanhas de marketing online que atingem muito mais pessoas por dia que uma loja física no melhor ponto de comércio do mundo, seja ele onde for. Isso acontece pois na internet não há barreiras geográficas, pode-se comprar e vender para qualquer lugar do mundo.

\begin{itemize}
  \item Facilidade e conforto de fazer compras sem sair de onde está.
  \item Maior disponibilidade de lojas e produtos.
  \item Melhores condições para pesquisa e comparação de preços.
  \item Sem filas ou espera por vendedores ou atendentes livres.
  \item Acesso a produtos de várias regiões do país e do mundo.
  \item Fácil acesso a promoções e cupons de desconto.
  \item Lojas abertas o tempo todo.
  \item 
\end{itemize}

\subsection{Desvantagens}

Nem tudo é perfeito no mundo do e-commerce. Salvo excessões, estas são algumas desvantagens deste tipo de negócio.

\begin{itemize}
  \item Está sujeito a falhas e pode ser vulnerável a ataques.
  \item Em transações com cartões de crédito, há o risco de roubo ou fraudes.
  \item Impossibilita a inspeção física do bem a ser adquirido.
  \item Preço e tempo de entrega podem inviabilizar o negócio.
  \item Dificuldade e demora no retorno ou troca de mercadorias
  \item Nem tudo pode ser vendido online, como por exemplo comidas perecíveis.
  \item Produtos de alto custo como jóias não tem segurança suficiente para serem despachados como encomenda.
  \item Não há o toque pessoal, interação entre cliente e comprador, que pode fazer diferença na concretização de uma venda.
\end{itemize}


% =========================================================================== %
\section{Ambiente de Desenvolvimento}

Existem várias maneiras de se criar um website, com várias linguagens e frameworks disponíveis. Linguagens como PHP, Ruby, Java e C# estão entre as linguagens mais utilizadas \cite{UsageStatistics}. Fazer um website simples do zero não é tarefa difícil. Porém no caso de um e-commerce, onde segurança e usabilidade são fatores chave para o sucesso do negócio, desenvolver em cima de camadas de software que já te proporciona muitas funcionalidades que serão essenciais, segurança contra ataques e falhas, além de uma comunidade de desenvolvedores que disponibiliza código e ajuda, faz desta a melhor opção. Mais adiante (!!!!!!!!colocar o capítulo!!!!!!!!), falaremos da escolha deste framework, quais eram as opções e o critério usado para a seleção.

Outra parte importante do desenvolvimento de um website é a escolha de um serviço de hosting. Servidores existem em várias partes do mundo e para sites de pequeno tráfego podemos encontrar por preços que vão de 100 reais por ano por um servidor compartilhado a 500 reais por ano uma maquina virtual exclusiva. A qualidade do servidor, sua conexão com a internet e localização física podem fazer muita diferença.

Como mais um fator a se considerar, temos o domínio. Domínio é o endereço que o usuário vai usar no browser para utilizar o website. Alguns domínios podem ser contratados em websites especializados de graça, como por exemplo o domínio `.tk` no `site www.dot.tk`. Estes domínios tem pouca visibilidade nos motores de busca e tem pouca confiança dos usuários por não serem amplamente utilizados. No Brasil, os domínios com melhor visibilidade e confiança são o `.com.br` e o `.com`, que por um custo não muito elevado, por volta de 50 reais, podem ser contratados por um ano.

É muito importante para o desenvolvedor de websites, garantir que o código que foi feito, funcione no servidor da mesma maneira que funcionou no ambiênte de desenvolvimento, computador pessoal ou notebook, de todos que participaram do projeto. Para não termos que instalar o mesmo sistema operacional e softwares em todos os computadores que forem utilizados, o uso de uma maquina virtual é a saida. Em uma maquina virtual, com as mesmas configurações do servidor que será utilizado e de fácil replicação nas máquinas de desenvolvimento, este problema é solucionado.

Outra parte esencial para o desenvolvimento de qualquer tipo software, principamente em web, são os softwares gerenciadores de pacotes e automatizadores de tarefas. Gerenciadores de pacotes ajudam na controle das bibliotecas que serão utilizadas e suas versões. Algumas delas já tomam conta de possíveis dependências que estas podem ter e também facilitam o upgrade ou downgrade quando necessários. Automatizadores de tarefas são úteis principalmente no front-end de uma aplicação web. Muitas tarefas como compilação, minificação e manipulação de arquivos podem ser feitas de forma automática, economizando tempo de desenvolvimento e ajudando o programador a focar no código.

% =========================================================================== %
\section{Tecnologias de Web}

Em aplicações web, temos código que executam tanto no servidor, quanto no cliente. O servidor é onde todas as informações estão guardadas em um banco de dados e o código que é executado nesta fase é chamado de back-end. Já no cliente, normalmente um browser, o código é geralmente para melhorar a experiência do usuário, mostrar as informações de forma mais clara, carregar informações dinamicamente sem mudar de página, fazer interações com o usuário e utilizar efeitos em elementos da página. Esta parte é chamada de front-end.


% =========================================================================== %
\section{SEO, Performance e Segurança}

\subsection{Search Engine Optimization}

Otimização para motores de busca (do inglês) são um conjunto de técnicas utilizadas em websites para melhorar o seu posicionamento em buscadores como o Google, Bing e Yahoo. Esta é uma parte essencial para a empresa que quer ter visibilidade na internet. Quanto mais próximo do topo em motores de busca, maior a chance de usuários clicarem no seu link e visitarem seu website. Hoje em dia existem empresas e proficionais especializados em SEO, dada a importância do assunto.

\subsection{Performance}

O tempo de carregamento e de renderização de um website impactam muito na quantidade de usuários que ele vai reter. Segundo Shaun Anderson \cite{LoadTime}, SEO da empresa MBSA Marketing LTD, cada segundo a mais que um website demora para carregar, a taxa de abandono aumenta cerca de 6\%, chegando a 25\% após 4 segundos. Além da perda de potenciais clientes, sites com performance probre tem seu rankeamento prejudicado em motores de busca. Existem várias ferramentas e softwares que testam a velocidade e dão dicas para melhora-la, como por exemplo o PageSpeed Insights do Google e o YSlow do Yahoo.

\subsection{Segurança}

Quando trabalhamos com cartões de crédito e informações confidenciais de clientes, temos que preocupar com a segurança destes dados. São quatro os principais fatores a se levar em consideração \cite{SecurityEcommerce}:

\begin{itemize}
  \item Privacidade: Todas as informações (!!!!!!!!!!!!!!!!!!)sensitivas dos clientes devem ser mantidas em bancos de dados seguros e longe do acesso de pessoas e empresas não autorizadas.
  \item Integridade: As informações dos clientes não devem ser modificadas ou adulteradas.
  \item Autenticação: Tanto o website quanto o cliente devem provar suas identidades um ao outro.
  \item Confirmação: É esperado que as informações trocadas sejam verificadas, para ter certeza que foram transmitidas e de forma correta.
\end{itemize}

% =========================================================================== %
\section{O Cliente}

A empresa Menina das Balas de Coco foi fundada 2010 na cidade de Itajubá, Minas Gerais. Seu produto principal e único é a bala de coco, doce típico brasileiro que costuma ser servido em festas de aniversário. O nome da empresa vem de como era conhecida a fundadora na faculdade, onde cursava Letras e vendia suas primeiras balas. A partir dai, foi-se aperfeiçoando as tecnicas de fabricação e melhorando o produto. Hoje a empresa tem mais de 20 sabores de balas que podem ter recheios diversos sabores, além da bala gelada e a original bala bombom, com cobertura de chocolate. Sua principal forma de vendas é por perfis em redes sociais.

