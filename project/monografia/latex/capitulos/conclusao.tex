\chapter{Conclusão}

Pela falta de interesse momentâneo do cliente, o site não foi divulgado e sua função e-commerce está desabilitada. Por este motivo não pudemos resultados relevantes coletados de SEO, que necessita um site minimamente divulgado. Apesar disto, mesmo sem um conteúdo relevante, aparecer na primeira página em uma busca mostra que estamos no caminho certo e que quando colocar-mos o site 100\% online com um conteúdo bem produzido e com algumas campanhas de marketing, ele será muito melhor indexado.

A performance obtida foi altamente satisfatória, acima do esperado e atingindo o objetivo definido. A surpresa fica pelo tempo de resposta do servidor, que apesar de ser compartilhado e de baixo custo, mostrou-se bem rápido e capaz. A ferramenta GTMetrix ajudou na medição e também com dicas nos vários teste que foram realizados durante o desenvolvimento, sendo de suma importância para o projeto.

O e-commerce, apesar de não abilitado, funcionou muito bem nos testes e é esperado que quando colocado a disposição do publico, funcione sem problemas. Até o momento não foi testada uma compra real, com um número de cartão válido, somente compras com cartões de testes providos pelo PayPal, que em teoria seriam suficientes. As funções de envio de e-mail ao final da compra foram testadas e funcionam sem problemas. No geral, o e-commerce funciona como esperado, mas ainda sim, tem muito a melhorar.

Como dito no capitulo anterior, um ponto que não teve um resultado satisfatório foi o design. Apesar disto, vejo que a dificuldade de ter este ponto re-feito é mínima e o mais importante é que está funcional e bem programado.

Finalmente, o projeto como um todo, teve um resultado satisfatório na maioria do pontos dados como objetivos, superando as espectatívas em alguns. Foi um objeto bastante complexo, por ter passar por várias áreas de um sistema web, como servidor, back-end, front-end e SEO, e por isso, bastante valioso para o aprendizado.