\chapter{Resultados}

\graphicspath{ {/var/www/html/meninadasbalas/projetc/monografia/latex/images/} }

% =========================================================================== %
\section{Desenvolvimento}
Para atingir os objetivos colocados e requerimentos do cliente, além de todo o processo de construção e configuração do site, tivemos que construir 2 módulos, adaptar versão de 1 e criar 1 sub-tema a partir de uma tema base.

\subsection{Módulo 1 - MBC Master}
O primeiro módulo foi chamado de MBC Master. Sua função é criar plugins de formatação de campos para o site. O campo de imagem de um produto pode ter carregadas até 10 imagens, porém em alguns casos como na lista de produtos da página inicial, apenas um imagem pode ser exibida. Para executar esta tarefa, foi copiado a classe de formatação do núcleo do Drupal para este módulo e modificada, para trazer apenas a primeira imagem. Por esta razão o plugin foi chamado de FirstImageFormatter.

\includegraphics{mbc_master}

\subsection{Módulo 2 - MBC Review}
Este módulo foi criado com o objetivo de prover um método para o administrador do site enviar e-mail para usuários pedindo Reviews de produtos ou Opiniôes sobre a empresa, prover um link exclusivo para criação destes conteúdos que vai neste e-mail, que reconhece automaticamente o usuário e não publique este conteúdo, deixando-o para moderação do administrador.

\subsection{Módulo Adaptado - TinyPNG}
Por ser um módulo bem simples que utiliza uma biblioteca externa para fazer a minificação, a adptação deste módulo foi bem simples. Uma função que é executada quando uma entidade (conteúdo) é salva foi utilizada e usada a biblioteca tinify \url{packagist.org/packages/tinify/tinify} para executar a minificação das imagens. Além disso, o módulo adiciona uma página administrativa para ser gerenciada a chave da API do serviço, que pode ser obtida no site \url{https://tinypng.com/}.

\section{Resultado II}

TODO

TODO Schema.org
