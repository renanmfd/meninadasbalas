% Resumo

Este é um projeto de um sistema web de e-commerce, construido para uma empresa real do ramo alimentício com base na cidade de Itajubá em Minas Gerais. O objetivo é criar um site rápido e seguro, que entregue ao usuário uma experiência única e apresente de forma clara e informativa o produto da empresa. Alguns conteúdos serão criados para enriquecer o site e chamar a atenção do cliente para o produto.

Utilizaremos o framework e gerenciador de conteúdos Drupal construido na linguagem PHP, escolhido por sua solidez e grande comunidade de desenvolvedores. Sua versão mais recente, o Drupal 8, é uma plataforma flexivel e inovadora, com sistemas de cache bem robustos e várias API's que fornecem métodos seguros de construir um website.

Nossos objetivos poderão ser atingidos com a ajuda de módulos da comunidade Drupal e as mais novas tecnologias e tecnicas de sistemas web que forem cabíveis. O fluxo de desenvolvimento será o mais seguro para a solidez do código, utilizando o software Vagrant para garantir ambiêntes constantes e git para controle de versão. O automatizador de tarefas Gulp e os gerenciadores de pacotes NPM e Composer serão algumas das ferramentas utilizadas para facilitar os processos de instalação e programação.

Esperamos que site tenha um desempenho acima da média e testes são feitos constantemente para garantir este quesito. A mais utilizada será o GTMetrix, ferramenta de teste de performance mais completa atualmente, que ainda dá dicas de pontos para  se melhorar.

\textbf{Palavras-chave}: e-commerce. drupal. web. php. performance. seo. twig. less.
